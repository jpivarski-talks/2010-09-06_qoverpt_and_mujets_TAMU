\documentclass[compress]{beamer}
\usepackage{ifthen,verbatim}

\newcommand{\isnote}{}
\xdefinecolor{lightyellow}{rgb}{1.,1.,0.25}
\xdefinecolor{darkblue}{rgb}{0.1,0.1,0.7}

%% Uncomment this to get annotations
%% \def\notes{\addtocounter{page}{-1}
%%            \renewcommand{\isnote}{*}
%% 	   \beamertemplateshadingbackground{lightyellow}{white}
%%            \begin{frame}
%%            \frametitle{Notes for the previous page (page \insertpagenumber)}
%%            \itemize}
%% \def\endnotes{\enditemize
%% 	      \end{frame}
%%               \beamertemplateshadingbackground{white}{white}
%%               \renewcommand{\isnote}{}}

%% Uncomment this to not get annotations
\def\notes{\comment}
\def\endnotes{\endcomment}

\setbeamertemplate{navigation symbols}{}
\setbeamertemplate{headline}{\mbox{ } \hfill
\begin{minipage}{5.5 cm}
\vspace{-0.75 cm} \small
\end{minipage} \hfill
\begin{minipage}{4.5 cm}
\vspace{-0.75 cm} \small
\begin{flushright}
\ifthenelse{\equal{\insertpagenumber}{1}}{}{Jim Pivarski \hspace{0.2 cm} \insertpagenumber\isnote/\pageref{numpages}}
\end{flushright}
\end{minipage}\mbox{\hspace{0.2 cm}}\includegraphics[height=1 cm]{../cmslogo} \hspace{0.1 cm} \includegraphics[height=1 cm]{../tamulogo} \hspace{0.01 cm} \vspace{-1.05 cm}}

\newcommand{\s}[1]{{\mbox{\scriptsize #1}}}

\begin{document}
%% \begin{notes}
%% \item This is the annotated version of my talk.
%% \item If you want the version that I am presenting, download the one
%% labeled ``slides'' on Indico (or just ignore these yellow pages).
%% \item The annotated version is provided for extra detail and a written
%% record of comments that I intend to make orally.
%% \item Yellow notes refer to the content on the {\it previous} page.
%% \item All other slides are identical for the two versions.
%% \end{notes}

\small

%% \begin{frame}
%% \frametitle{Outline}
%% \begin{itemize}\setlength{\itemsep}{0.75 cm}
%% \item 
%% \end{itemize}
%% %% \hspace{-0.83 cm} \textcolor{darkblue}{\Large Outline2}
%% \end{frame}

%% \section*{First section}
%% \begin{frame}
%% \begin{center}
%% \Huge \textcolor{blue}{First section}
%% \end{center}
%% \end{frame}

\begin{frame}
\frametitle{Schedule for a new alignment}
\framesubtitle{Target: beginning of November.  Which of these are possible in 2 months?}
\begin{itemize}
\item Endcap disk alignment with collisions muons
\begin{itemize}
\item study of residuals versus momentum (partly done)
\item understand fitter (the differences we see ought to be statistical: what's wrong with the statistics of the fit?)
\item currently waiting for re-processed collisions dataset\ldots
\end{itemize}

\item Chamber alignment with low-momentum cosmics
\begin{itemize}
\item study of residuals versus momentum
\item should add $|p|$-dependence to the fit: Gaussian $\sigma \to \sigma/|p|$
\end{itemize}

\item Chamber alignment with low-momentum collisions
\begin{itemize}
\item need at least $\mathcal{O}(10\mbox{ pb}^{-1})$ (not far in the future\ldots)
\item study of low-$|p|$ cosmics residuals vs.\ low-$|p|$ collisions residuals
\item compare collisions alignment in endcap with beam-halo and disk-shifts: that is, CSC chamber-level alignment ought to agree with the prior geometry within quoted uncertainties
\end{itemize}

\item Understand residuals inside of a chamber: if it's due to trigger
  bias and trigger bias can only be excluded by asking for a second
  muon, we'll need {\it a lot} more data to do a standard alignment
\end{itemize}
\label{numpages}
\end{frame}

\end{document}
